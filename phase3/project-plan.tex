% Be sure to address the following items:
% -   Introduction
% -   Define sequence of tasks to be performed
% -   Identify all deliverables (Milestone Deliveries)  - IMPORTANT:  This is a critical item in that your future deliveries will be evaluated based on this.   It needs to be clear, detailed, and objective (can I go through your milestone description like a check list to see if your delivery is complete?).   Note, as the project goes forward this can change, and that is perfectly OK.
% -   Define the dependency relationship between each task
% -   Estimate resources required to perform each task
% -   Schedule all tasks to be performed
% -   Define the organization executing the project
% -   Identify the known project risks (Risk Analysis)
% -   Define the process to ensure quality (Test Plan)
% -   Define the process for configuration management
% -   Define the process specifying and controlling the design requirements


\documentclass[12pt,titlepage]{article}
\usepackage[margin=1.0in]{geometry}
\usepackage{graphicx}
\usepackage[pdftex]{hyperref}
\usepackage[english]{babel}
\usepackage{csquotes}
\usepackage{titling}
\usepackage{titlesec}
\usepackage{tabularx}
\usepackage{float}
\usepackage{xspace}

\newcommand\tab[1][.5in]{\hspace*{#1}}
\newcommand\gametitle{\textit{\AE on Chronicles}}


\begin{document}

\section{Introduction}

This document outlines the project management guidelines for Team Epsilon's
development of \gametitle, an adventure RPG video game.

The rest of the document is organized as follows. TODO

\section{Tasks}

\subsection{Milestone Deliveries}

\subsection{Dependencies \& Resources}

\subsection{Task Schedule}

\section{Organization}

\section{Development}

\subsection{Documents}

The following is a list of documents relevant to maintaining the project.

\begin{table}[H]
\centering
\caption{Development Documents}
\begin{tabularx}{\linewidth}{|l|X|}
\hline
{\bf Document}         & {\bf Purpose} \\ \hline
{\it Design Document}  & \\ \hline
{\it Bugs \& Features} & Documents all upcoming and completed features, as well
                         as descriptions and discovery-dates of all bugs for a
                         particular feature. A status is provided for each
                         feature which indicates whether the feature has been
                         tested by the Test Lead for further bugs. Each bug
                         includes a Resolution field that indicates whether the
                         bug has been resolved, as well as a description as to
                         how the bug was resolved. \\ \hline
\end{tabularx}
\end{table}


\subsection{Risk Analysis}

\subsection{Test Plan}

In general, periodic plays and replays of sections of the game should be done to
provide a loose redundant check on the game's stability. A more formal testing
structure is defined next.

Testing of the game will take place in incremental steps throughout development
of the game. While automated testing would present a robust and consistent
testing method, the time constraints of the project and its nature as a video
game make this an infeasible testing strategy. Therefore, the team will rely on
a system of collaborative manual testing that works as follows:

\begin{enumerate}
\item When adding a new feature to the game, the developer(s) adding the feature
      will test it and ensure that all directly affected aspects of the game
      perform in the expected manner. The feature will then be documented in the
      {\it Bugs \& Features} document. Any remaining bugs that the developer is
      unable to resolve should also be documented in the {\it Bugs \& Fixtures}
      document.
\item It is then up to the Test Lead to monitor the {\it Bugs \& Features}
      document for new features that they should test for completeness and bugs.
      If the Test Lead is unable to resolve all bugs for any reason, they may
      delegate specific bug resolutions to other members of the team.

      {\it Caveat}: In the case that the Test Lead has added a particular
      feature themselves, they must act in accordance with step 1. The task of
      further testing (step 2) then falls on the Team Leader by default, though
      the Test Lead may assign this to another team member at their own
      discretion on a case-by-case basis.
\end{enumerate}

\subsection{Configuration Management}

\subsection{Design Requirement Update Procedure}

\end{document}
