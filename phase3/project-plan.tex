% Be sure to address the following items:
% -   Introduction
% -   Define sequence of tasks to be performed
% -   Identify all deliverables (Milestone Deliveries)  - IMPORTANT:  This is a
%     critical item in that your future deliveries will be evaluated based on
%     this.   It needs to be clear, detailed, and objective (can I go through
%     your milestone description like a check list to see if your delivery is
%     complete?).   Note, as the project goes forward this can change, and that
%     is perfectly OK.
% -   Define the dependency relationship between each task
% -   Estimate resources required to perform each task
% -   Schedule all tasks to be performed
% -   Define the organization executing the project
% -   Identify the known project risks (Risk Analysis)
% -   Define the process to ensure quality (Test Plan)
% -   Define the process for configuration management
% -   Define the process specifying and controlling the design requirements




\documentclass[12pt,titlepage]{article}
\usepackage[margin=1.0in]{geometry}
\usepackage{graphicx}
\usepackage[pdftex]{hyperref}
\usepackage[english]{babel}
\usepackage{csquotes}
\usepackage{titling}
\usepackage{titlesec}
\usepackage{tabularx}
\usepackage{float}
\usepackage{xspace}
\usepackage{tabularx}
\usepackage{parskip}

\usepackage{enumitem,amssymb}
\newlist{todolist}{itemize}{2}
\setlist[todolist]{label=$\square$}
\setlength\parindent{0em}


\newcommand\tab[1][.5in]{\hspace*{#1}}
\newcommand\gametitle{\textit{\AE on Chronicles}\xspace}

\title{Project Plan}
\author{Team Epsilon}
\date{\today}

\begin{document}
\maketitle

\section{Introduction}

This document outlines the project management guidelines for Team Epsilon's
development of \gametitle, an adventure RPG video game.

The rest of the document is organized as follows. Section~\ref{sec:tasks}
outlines the tasks that will be performed at each milestone, what dependencies
exist between the tasks, and what the task schedule will be.
Section~\ref{sec:org} explains the organization of the team.
Section~\ref{sec:dev} outlines the development process including what documents
will be maintained and a test plan to ensure game quality.

\section{Tasks}
\label{sec:tasks}
This section outlines the tasks that must be completed to ensure that the
development of \gametitle is successful.

\subsection{Milestone Deliveries}
\begin{todolist}
    \item \textbf{Milestone I:} Overworld
        \begin{todolist}
            \item Character Movement
            \item NPC interaction
            \item Combat Hooks
            \item Menu UI Systems
        \end{todolist}

    \item \textbf{Milestone II:} Combat System
        \begin{todolist}
            \item Deck Randomization
            \item Turn Based System
            \item Combat Animations
            \item End of combat (apply damage)
        \end{todolist}

    \item \textbf{Milestone III:} Alpha Release
        \begin{todolist}
            \item Integrate Overworld and combat
            \item Story flushed out
            \item A working location
            \item Sound Assets
        \end{todolist}

    \item \textbf{Milestone IV:} Beta Release
        \begin{todolist}
            \item All regions of world
            \item Asset Integration
            \item Pool of approximately 30 cards per character
            \item 30 non-character specific cards
            \item Minimum of 4 elemental status types
            \item Card Crafting
            \item Music
        \end{todolist}

    \item \textbf{Milestone V:} Final Release
        \begin{todolist}
            \item bug fixes
            \item stretch goals
        \end{todolist}
\end{todolist}

\subsection{Dependencies \& Resources}
Table~\ref{tab:dependencies} outlines the major dependencies between the tasks
in our project.
\begin{table}[H]
    \caption{Dependencies}
    \label{tab:dependencies}
    \centering
    \begin{tabularx}{\linewidth}{|l|X|}
        \hline
        \textbf{Task} & \textbf{Dependent Tasks} \\
        \hline
        Character Movement in Overworld & NPC Interaction, Combat Hooks \\
        \hline
        Turn Based Combat System & End of Combat (apply damage), Overworld and
        Combat Integration \\
        \hline
        Card Design & Card Enhancement behavior via Elements Mechanic, Card Crafting \\
        \hline
        Flesh Out Story & A working location \\
        \hline
    \end{tabularx}
\end{table}

\subsection{Task Schedule}
The following table shows the timetable for milestone completion.
\begin{table}[H]
    \caption{Milestone Delivery Schedule}
    \label{tab:schedule}
    \centering
    \begin{tabular}{|l|l|}
        \hline
        \textbf{Milestone} & \textbf{Date} \\
        \hline
        Milestone I: Overworld & 2017/02/27 \\
        \hline
        Milestone II: Combat & 2017/03/13 \\
        \hline
        Milestone III: Alpha & 2017/03/24 \\
        \hline
        Milestone IV: Beta & 2017/04/14 \\
        \hline
        Milestone V: Final & 2017/04/26 \\
        \hline
    \end{tabular}
\end{table}

\section{Organization}
\label{sec:org}
Each team member has a specific set of tasks described below. In addition to
their individual tasks, all team members will contribute to the coding effort.
\begin{itemize}
    \item \textbf{Caleb:} Recorder and Scheduler, Programming, Test Lead
    \item \textbf{David:} Programming, Story Creation
    \item \textbf{Jacob:} Programming, Sound \& Music
    \item \textbf{Robbie:} Team Lead, Concept Art, Programming
    \item \textbf{Sumner:} Programming, Document Maintenance
\end{itemize}

\section{Development}
\label{sec:dev}

\subsection{Documents}

The following is a list of documents relevant to maintaining the project.

\begin{table}[H]
\centering
\caption{Development Documents}
\begin{tabularx}{\linewidth}{|l|X|}
\hline
{\bf Document}         & {\bf Purpose} \\ \hline
{\it Design Document}  & \\ \hline
{\it Bugs \& Features} & Documents all upcoming and completed features, as well
                         as descriptions and discovery-dates of all bugs for a
                         particular feature. A status is provided for each
                         feature which indicates whether the feature has been
                         tested by the Test Lead for further bugs. Each bug
                         includes a Resolution field that indicates whether the
                         bug has been resolved, as well as a description as to
                         how the bug was resolved. \\ \hline
\end{tabularx}
\end{table}


\subsection{Risk Analysis}

\subsection{Test Plan}

In general, periodic plays and replays of sections of the game should be done to
provide a loose redundant check on the game's stability. A more formal testing
structure is defined next.

Testing of the game will take place in incremental steps throughout development
of the game. While automated testing would present a robust and consistent
testing method, the time constraints of the project and its nature as a video
game make this an infeasible testing strategy. Therefore, the team will rely on
a system of collaborative manual testing that works as follows:

\begin{enumerate}
\item When adding a new feature to the game, the developer(s) adding the feature
      will test it and ensure that all directly affected aspects of the game
      perform in the expected manner. The feature will then be documented in the
      {\it Bugs \& Features} document. Any remaining bugs that the developer is
      unable to resolve should also be documented in the {\it Bugs \& Fixtures}
      document.
\item It is then up to the Test Lead to monitor the {\it Bugs \& Features}
      document for new features that they should test for completeness and bugs.
      If the Test Lead is unable to resolve all bugs for any reason, they may
      delegate specific bug resolutions to other members of the team.

      {\it Caveat}: In the case that the Test Lead has added a particular
      feature themselves, they must act in accordance with step 1. The task of
      further testing (step 2) then falls on the Team Leader by default, though
      the Test Lead may assign this to another team member at their own
      discretion on a case-by-case basis.
\end{enumerate}

\subsection{Configuration Management}

\subsection{Design Requirement Update Procedure}

If any member of the team wants to change the design, the following steps must
be followed.

\begin{enumerate}
    \item The person who wants to change the design will notify the rest of the
        team and explain the change.

    \item If there is immediate consensus, or if it is a minor change, the
        person who proposed the change can proceed with the change (see 4).

    \item If there is not immediate consensus, a vote will be held and if a
        majority of the team members approve the change, the person who proposed
        the change can proceed with the change.

    \item After the change is approved by the team, the person who proposed the
        change has the responsibility of updating the design document with said
        change or delegating the task to another willing team member.

    \item After the Design Document has been updated, the Document Maintainer
        (Sumner) will be notified and he will ensure that the change fits the
        document organization.
\end{enumerate}

\end{document}
