\documentclass{teamepsilon}

\title{{\AE}on Chronicles}
\author{Team Epsilon}
\institute{Colorado School of Mines}

% The pitch proposal, sometimes known as a concept document, is typically a
% short document or PowerPoint designed to communicate the game concept and sell
% it to the publisher or investors.  In the real world this may run from 10-20
% pages but for this project and the given short time frame, the team need not
% prepare a full blown proposal. However, it should include basic technical
% details as well like how the team will build the game (languages, libraries,
% game engines, etc.  and team responsibilities). The pitch should run 5-6
% minutes.

\begin{document}

\begin{frame}{Overview}

\end{frame}

\begin{frame}{Gameplay}

\end{frame}

\begin{frame}{Art}

\end{frame}

\begin{frame}{Technical Details}
    The Team Epsilon development team will use the following tools to build
    \textit{{\AE}on Chronicles}:

    \begin{itemize}
        \item \textbf{GameMaker} game engine
        \item \textbf{Photoshop} for creating electronic artwork
        \item \textbf{Visual Studio} for implementing game physics and other
            complex logic
    \end{itemize}

    The team will use \textbf{Game Maker Language} and \textbf{C++} for the game
    code.
\end{frame}

\begin{frame}{Team Responsibilities}
    \begin{itemize}
        \item \textbf{Caleb:}
        \item \textbf{David:}
        \item \textbf{Jacob:}
        \item \textbf{Robbie:}
        \item \textbf{Sumner:}
    \end{itemize}
\end{frame}

\end{document}
