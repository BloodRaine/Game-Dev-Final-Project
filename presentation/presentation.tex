\documentclass{teamepsilon}

\title{GAME NAME}
\author{Team Epsilon}
\institute{Colorado School of Mines}

% The pitch proposal, sometimes known as a concept document, is typically a
% short document or PowerPoint designed to communicate the game concept and sell
% it to the publisher or investors.  In the real world this may run from 10-20
% pages but for this project and the given short time frame, the team need not
% prepare a full blown proposal. However, it should include basic technical
% details as well like how the team will build the game (languages, libraries,
% game engines, etc.  and team responsibilities). The pitch should run 5-6
% minutes.

\begin{document}

\begin{frame}{Overview}

\end{frame}

\begin{frame}{Gameplay}
    \begin{itemize}
        \item Top-down overworld, much like original \textit{Final Fantasy}
        or \textit{Pok\'emon} games
        \item Combat system will be turn-based, but will use a deck of in-game
        cards to create attack and defense combos
        \item Cards will also be used for items such as healing potions,
        power-ups, et al.
    \end{itemize}
\end{frame}

\begin{frame}{Gameplay Concept}
Overworld \hfill Combat \\
TODO add concept art of overworld and in combat
\end{frame}

\begin{frame}{Art}

\end{frame}

\begin{frame}{Technical Details}

\end{frame}

\begin{frame}{Team Responsibilities}
    \begin{itemize}
        \item \textbf{Caleb:} Recorder and scheduler, help with artwork,
        programming
        \item \textbf{David:} Programming, story creation
        \item \textbf{Jacob:} Programming, sound \& music
        \item \textbf{Robbie:} Team lead, concept art, programming
        \item \textbf{Sumner:} Programming, document maintenance
    \end{itemize}
\end{frame}

\end{document}
